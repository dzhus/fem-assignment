\documentclass{article}
\usepackage[utf8x]{inputenc}
\usepackage[english,russian]{babel}
\usepackage{amsmath,amssymb}
\usepackage{misccorr}

\usepackage{tikz}
\usetikzlibrary{calc,decorations.pathreplacing}
\tikzset{dot/.style={circle,fill=black,scale=0.5}}

\newcommand{\figref}[1]{рис. \ref{#1}}
\newcommand{\node}[1]{$#1$}
\begin{document}

\begin{figure}[hb]
  \centering
  \begin{tikzpicture}
    \node [dot, label=above right:\node{1}] (1) at (0, 0) {};
    \node [dot, label=above left:\node{2}] (2) at ($ (1)-(0,3) $) {};
    \node [dot, label=above right:\node{3}] (3) at ($ (2)-(0,3) $) {};
    \node [dot, label=above right:\node{4}] (4) at ($ (3)-(0,3) $) {};
    \draw (1) -- (2) -- (3) -- (4);
    \draw [thick] ($ (4) - (1, 0) $) -- ($ (4) + (1, 0) $);
    
    \draw ($ (1)+(0, 1) $) node[right] {$\alpha$} arc (90:120:1);

    \draw[dashed] (1) -- ($ (1) + (0, 1.5) $);

%    \draw (2) -- ++(0.3, 0.3) -- ++(0, -0.6) -- (2);
%    \draw (3) -- ++(-0.3, 0.3) -- ++(0, -0.6) -- (3);

    \draw [>=latex, ->] ($ (1)+(-0.5, 1) $) node[above] {$P$} -- (1) ;
  \end{tikzpicture}
  \label{fig:stand}
\end{figure}

Рассмотрим \figref{fig:stand}, на котором изображена конструкция из
трёх балочных элементов. К верхнему узлу приложена сила $P$, вектор
которой направлен под малым углом к оси балок. Нижний
узел имеет шарнирное закрепление.

Параметры элементов: длина $L = 1$ м, площадь сечения $A =
10^{-4}$ м², момент инерции сечения $I = 8.3 \times 10^{-10}$ м⁴,
модуль упругости $E = 70$ Гпа. Модуль упругости взят как для
алюминия.

Матрица жёсткости балочного элемента имеет вид:
\begin{equation*}
  k = \left(
    \begin{smallmatrix}
      AE/L& 0& 0& -AE/L& 0& 0\\
      0& 12EI/L^3& 6EI/L^2& 0& -12EI/L^3& 6EI/L^2\\
      0& 6EI/L^2& 4EI/L& 0& -6EI/L^2& 2EI/L\\
      -AE/L& 0& 0& AE/L& 0& 0\\
      0& -12EI/L^3& -6EI/L^2& 0& 12EI/L^3& -6EI/L^2\\
      0& 6EI/L^2& 2EI/L& 0& -6EI/L^2& 4EI/L
    \end{smallmatrix}\right)
\end{equation*}

Каждый из узлов элемента имеет три степени свободы: угловое и
поперечное смещения $u, v$, а также $\theta$ — угол изгиба вокруг оси
$z$

Глобальная матрица жёсткости для системы имеет размерность $12 \times 12$.
\begin{equation*}
  K = \left(
  \begin{smallmatrix}
    \frac{AE}{L}& 0& 0& -\frac{AE}{L}& 0& 0& 0& 0& 0& 0& 0& 0\\
     0& 12\frac{EI}{L^3}& 6\frac{EI}{L^2}& 0& -12\frac{EI}{L^3}& 6\frac{EI}{L^2}& 0& 0& 0& 0& 0& 0\\
     0& 6\frac{EI}{L^2}& 4\frac{EI}{L}& 0& -6\frac{EI}{L^2}& 2\frac{EI}{L}& 0& 0& 0& 0& 0& 0\\
     -\frac{AE}{L}& 0& 0& 2\frac{AE}{L}& 0& 0& -\frac{AE}{L}& 0& 0& 0& 0& 0\\
     0& -12\frac{EI}{L^3}& -6\frac{EI}{L^2}& 0& 24\frac{EI}{L^3}& 0& 0& -12\frac{EI}{L^3}& 6\frac{EI}{L^2}& 0& 0& 0\\
     0& 6\frac{EI}{L^2}& 2\frac{EI}{L}& 0& 0& 8\frac{EI}{L}& 0& -6\frac{EI}{L^2}& 2\frac{EI}{L}& 0& 0& 0\\
     0& 0& 0& -\frac{AE}{L}& 0& 0& 2\frac{AE}{L}& 0& 0& -\frac{AE}{L}& 0& 0\\
     0& 0& 0& 0& -12\frac{EI}{L^3}& -6\frac{EI}{L^2}& 0& 24\frac{EI}{L^3}& 0& 0& -12\frac{EI}{L^3}& 6\frac{EI}{L^2}\\
     0& 0& 0& 0& 6\frac{EI}{L^2}& 2\frac{EI}{L}& 0& 0& 8\frac{EI}{L}& 0& -6\frac{EI}{L^2}& 2\frac{EI}{L}\\
     0& 0& 0& 0& 0& 0& -\frac{AE}{L}& 0& 0& \frac{AE}{L}& 0& 0\\
     0& 0& 0& 0& 0& 0& 0& -12\frac{EI}{L^3}& -6\frac{EI}{L^2}& 0& 12\frac{EI}{L^3}& -6\frac{EI}{L^2}\\
     0& 0& 0& 0& 0& 0& 0& 6\frac{EI}{L^2}& 2\frac{EI}{L}& 0& -6\frac{EI}{L^2}& 4\frac{EI}{L}
  \end{smallmatrix}\right)
\end{equation*}

Уравнение для нахождения смещений узлов конструкции имеет вид:
\begin{equation*}
  \begin{bmatrix}
    K_{aa}& K_{ac} \\
    K_{ca}& K_{cc}
  \end{bmatrix}
  \begin{bmatrix}
    U_a\\
    U_c\\
  \end{bmatrix}
  =
  \begin{bmatrix}
    F_a\\
    F_c
  \end{bmatrix}.
\end{equation*}

Здесь:
\begin{itemize}
\item $U_c$ — вектор заданных перемещений. В нашем случае он состоит
  из двух нулей в силу закрепления четвёртого узла.

\item $F_a$ — вектор приложенных внежних нагрузок. В нашем случае он
  имеет вид:
  \begin{equation*}
    F_a =
    \begin{bmatrix}
      P_y\\
      P_x\\
      0\\
      0\\
      0\\
      0\\
      0\\
      0\\
      0\\
      0\\
    \end{bmatrix},
  \end{equation*}
  где $P_x, P_y$ — проекции силы $P$ на вертикальную и горизонтальную
  оси (в глобальной системе координат). Вектор имеет десять компонент.

\item $U_a$ — вектор неизвестных перемещений. Находить его будем по
  формуле
  \begin{equation}
    \label{eq:solving}
    U_a = {K_{aa}}^{-1}({F_a} - {K_{ac}}{U_c})
  \end{equation}
\end{itemize}

Матрицы $K_{aa}$ и $K_{ac}$ состоят из строк матрицы $K$ 1--9 и 12,
соответственно неизвестным степеням свободы системы.

Построим зависимость формы стержня от угла $\alpha$ и силы $P$.
\begin{figure}\centering
    \begin{tikzpicture}[scale=5]
\draw[thick] (0.133505, -0.000214) -- (0.058408, -1.000143) -- (0.012516, -2.000071) -- (0.000000, -3.000000);
\draw[>=latex, ->] ($ (0.133505, -0.000214) + (91.000000:0.250000) $) -- (0.133505, -0.000214);

\draw[thick] (0.666711, -0.000213) -- (0.291686, -1.000142) -- (0.062504, -2.000071) -- (0.000000, -3.000000);
\draw[>=latex, ->] ($ (0.666711, -0.000213) + (95.000000:0.250000) $) -- (0.666711, -0.000213);

\draw[thick] (1.328347, -0.000211) -- (0.581152, -1.000141) -- (0.124533, -2.000070) -- (0.000000, -3.000000);
\draw[>=latex, ->] ($ (1.328347, -0.000211) + (100.000000:0.250000) $) -- (1.328347, -0.000211);

\draw[thick] (1.979874, -0.000207) -- (0.866195, -1.000138) -- (0.185613, -2.000069) -- (0.000000, -3.000000);
\draw[>=latex, ->] ($ (1.979874, -0.000207) + (105.000000:0.250000) $) -- (1.979874, -0.000207);

\draw[dashed] (0, 0) -- (0, -3);
  \end{tikzpicture}
  \caption{Положение узлов конструкции после приложения силы в 500 Н
    под углами 1°, 5°, 10° и 15°}
\end{figure}

\begin{figure}\centering
    \begin{tikzpicture}[scale=5]
\draw[thick] (0.267009, -0.000429) -- (0.116817, -1.000286) -- (0.025032, -2.000143) -- (0.000000, -3.000000);
\draw[>=latex, ->] ($ (0.267009, -0.000429) + (91.000000:0.500000) $) -- (0.267009, -0.000429);

\draw[thick] (1.333421, -0.000427) -- (0.583372, -1.000285) -- (0.125008, -2.000142) -- (0.000000, -3.000000);
\draw[>=latex, ->] ($ (1.333421, -0.000427) + (95.000000:0.500000) $) -- (1.333421, -0.000427);

\draw[thick] (2.656694, -0.000422) -- (1.162304, -1.000281) -- (0.249065, -2.000141) -- (0.000000, -3.000000);
\draw[>=latex, ->] ($ (2.656694, -0.000422) + (100.000000:0.500000) $) -- (2.656694, -0.000422);


\draw[dashed] (0, 0) -- (0, -3);
  \end{tikzpicture}
  \caption{Положение узлов конструкции после приложения силы в 1 кН
    под углами 1°, 5°, 10°}
\end{figure}
\clearpage
Предположим теперь, что в узле два находится ограничитель,
накладывающий ограничение $v_2 \leq 0$ на поперечное смещение второго
узла. Представить это при решении можно с помощью итерационного
метода, заключающегося в постепенном увеличении внешней силы $R_2$,
приложенной к узлу 2 и направленной поперёк балки, до тех пор, пока в
решении $v_2$ не перестанет быть положительным.
\begin{figure}\centering
    \begin{tikzpicture}[scale=3]
\draw[red, thick] (-1.000103, -0.000427) -- (-0.000009, -1.000285) -- (0.125008, -2.000142) -- (0.000000, -3.000000);
\draw[>=latex, ->] ($ (-1.000103, -0.000427) + (95.000000:0.500000) $)  node[right] {1 кН}
-- (-1.000103, -0.000427);
\draw[>=latex, ->] ($ (-0.000009, -1.000285) + (000000:0.0915) $)
node[above right] {183.03 Н} -- (-0.000009, -1.000285);

\draw[] (1.333421, -0.000427) -- (0.583372, -1.000285) -- (0.125008, -2.000142) -- (0.000000, -3.000000);
\draw[>=latex, ->] ($ (1.333421, -0.000427) + (95.000000:0.500000) $)
-- (1.333421, -0.000427);

\draw[dashed] (0, 0) -- (0, -3);
  \end{tikzpicture}
  \caption{Моделирование ограничений в промежуточных узлах. Внешняя
    нагрузка 1 кН, $\alpha=5°$}
\end{figure}




\end{document}
